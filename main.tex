\documentclass[titlepage]{article}
\usepackage{graphicx}
\usepackage{fancyhdr}
\usepackage{hyperref} % Required for inserting images

\title{CW-final-project}
\author{Pooriya Jafari}
\date{January 2024}

\pagestyle{fancy}


\begin{document}
\maketitle
\tableofcontents

\newpage

\section{Git and GitHub}
\subsection{Repository Initialization and Commits}
at first I created a repository on GitHub and then I cloned it on my local machine. then I created a file named \textbf{README.md} and added some text to it. then I added the file to the staging area and committed it. then I pushed the changes to the remote repository.

\subsection{GitHub Actions for LaTeX Compilation}
At first, I copied the files in the repositories in the document, but I noticed that the things that were supposed to happen did not happen. After that, I realized that it should happen by adding action tags and release the existing PDF. So, I added the tag to the parts I wanted with the commands available on the internet


\newpage

\section{Exploration Tasks}
\subsection{Vim Advanced Features}
\begin{enumerate}
    \item Regular Expressions and Search Patterns\\
    Vim supports powerful regular expressions for search and replace operations. You can create complex search patterns to find and manipulate text efficiently.

    \item Registers and Shell Integration\\
    Vim's registers not only store yanked or deleted text but also allow you to interact with the system shell. You can execute shell commands from within Vim, capture their output, and even paste the output directly into your document.

    \item Autocompletion and IntelliSense\\
    Vim supports autocompletion for code, making coding faster and more accurate. Plugins like YouCompleteMe or coc.nvim provide IntelliSense-like features, offering suggestions for code completion, function signatures, and documentation as you type.
\end{enumerate}
\subsection{Memory profiling}

\subsubsection{Memory Leak}
When we store more information than the capacity allocated to the memory
\subsubsection{Memory profilers}
Valgrind is a tool for finding memory-related errors in programs. It helps identify issues like memory leaks. One of its tools, Memcheck, is commonly used for detecting memory leaks by tracking memory allocations and deallocations during program execution. Developers run Valgrind to analyze their programs and get detailed reports on memory-related problems.
\newpage
\subsection{GNU/Linux Bash Scripting}
\subsubsection{fzf}
\begin{enumerate}
    \item Fuzzy searching finds approximate matches for a query, accommodating slight errors or variations in the input data. It's useful for applications like spell checking and search engines.

    \item The command `ls | fzf` uses the `ls` command to list files in a directory and pipes the output to `fzf` for interactive fuzzy searching and selection.
\end{enumerate}
\subsubsection{Using fzf to find your favorite PDF}
\begin{verbatim}
fd -e pdf
fd -e pdf | fzf
\end{verbatim}
\subsubsection{Opening the file using Zathura}
\begin{verbatim}
zathura "$(fd -e pdf | fzf)"
\end{verbatim}
\section{Git and FOSS}
\subsection{README.md}
done. you can see that in the repository.

\subsection{Issues}
\begin{figure}[h]
    \centering
    \includegraphics[width=1\textwidth]{images/img1.png}
    \caption{Issues}
\end{figure}
\end{document}

